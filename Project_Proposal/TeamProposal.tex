\documentclass[titlepage]{article}

\usepackage{tabularx}
\usepackage{booktabs}
\usepackage[bottom=3cm, right=3cm, left=3cm, top=3cm]{geometry}
\usepackage{graphicx}
\usepackage{hyperref}
\usepackage{float}

\title{COMPSCI 4NL3: Natural Language Processing\\Team Proposal}

\author{Team 4\\
\\ Junnan Li
\\ Nawaal Fatima
\\ Rashad Bhuiyan
\\ Sumanya Gulati}                  

\date{24 January 2025}

\begin{document}

\begin{titlepage}
    \maketitle
\end{titlepage}

\newpage 

\tableofcontents
\listoftables
\listoffigures

\newpage

% Note that I am copying over the sections just as they are mentioned in the 
% instructions document. Feel free to change them up as you want.

\section{Task Overview --Item 2}
% Include title, significance and what makes it challenging.

\section{Task Definition --Item 3}
% Type of data, classification or regression, number of classes, single label
% or multi-label.

\section{Data Source and Plan for Data Collection --Item 4}
% May include how we are going to scrape the data and follow terms-of-service,
% API access and handling rate limiting, using open source data or any other 
% relevant details.

\section{Dataset Details --Item 5}
\indent A data point will be considered a single paragraph of a response to an interview question by a player or coach. 
For each event, we will look at each and every day of interview recordings that are tracked. The events covered 
this time will be the entirety of both NBA and WNBA Finals, as well as both NBA and WNBA Drafts. Based on these 
events, there are a total of 17 days of interviews. Furthermore, each recording day has interviews from multiple 
different players and coaches that each answer questions with either a one or multi-paragraph response. Based on 
this information, our dataset is expected to contain approximately 2500 data points.
\newline \newline
\indent The following is a set of 3 data points taken from the Game 2 Postgame interview with Jason Kidd of the Dallas 
Mavericks:
\begin{enumerate}
    \item Yeah, we are not down. We're positive. This is a group that believes. We didn't get an opportunity to get a 
    split or win two here on the road. Now Boston held serve. Now we've got to go home and hold serve.
    \begin{itemize}
        \item \textbf{Assigned Labels: Positive, Team}
    \end{itemize}

    \item Big. The small things, you know, we have to do the small things, and that's part of the game. Those are points 
    that we left on the board, and we didn't shoot free throws well tonight, and we have to be better.
    \begin{itemize}
        \item \textbf{Assigned Labels: Negative, Team}
    \end{itemize}

    \item Yeah, I think Luka is a special player. He's one of, if not the best player in the world, and he causes a problem. 
    He's able to find guys. Again, creating open opportunities, and we just didn't take advantage of it.
    \begin{itemize}
        \item \textbf{Assigned Labels: Positive, Individual}
    \end{itemize}
\end{enumerate}
\section{Team Contract --Item 6}
% Description of the expectations and/or roles that team members agree upon 
% for the semester. Part of final project grade will be based on whether 
% the teammates agree that we followed this contract. It should specify the
% overall purpose of the team, the responsibilities and the ground rules
% the team members agree to follow.

\subsection{Team Purpose}
% What is our team's purpose or mission?

\subsection{Team Member Roles}
% What are the duties/roles of each team member? What is expected of each 
% team member?

\subsection{Facilitation Activities}
% How will the team handle the leadership/facilitation/management activities?

\subsection{Anything Else???}
% Anything else we think will be helpful to set the group expectations.
% Print name, sign and date.

\end{document}