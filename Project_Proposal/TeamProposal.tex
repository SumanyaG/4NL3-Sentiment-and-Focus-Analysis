\documentclass[titlepage]{article}

\usepackage{tabularx}
\usepackage{booktabs}
\usepackage[bottom=3cm, right=3cm, left=3cm, top=3cm]{geometry}
\usepackage{graphicx}
\usepackage{hyperref}
\usepackage{float}
\newcommand\tab[1][1cm]{\hspace*{#1}}

\title{COMPSCI 4NL3: Natural Language Processing\\
Team Proposal}

\author{Team 4\\
\\ Junnan Li
\\ Nawaal Fatima
\\ Rashad Bhuiyan
\\ Sumanya Gulati}                  

\date{24 January 2025}

\begin{document}

\begin{titlepage}
  \maketitle
\end{titlepage}

\newpage 

\tableofcontents
\listoftables

\newpage

\section{Task Overview}
% Include title, significance and what makes it challenging.
This project, titled \textbf{`Sentiment and Focus Analysis in Post-Game Interviews'} aims to analyze post-game 
interview responses from players and coaches to gain insights into their communication styles and focus areas.
By examining these responses, we seek to classify each statement along two axes:

\begin{enumerate}
  \item Whether the response focuses on the \emph{individual (self)} or the \emph{team}.
  \item Whether the sentiment expressed is \emph{positive} or \emph{negative}.
\end{enumerate}

This analysis will shed light on how players and coaches emphasize teamwork or individual performance and how their
sentiments vary depending on the context. The results can contribute to a deeper understanding of communication
patterns within sports and their alignment with team dynamics and morale.\\

The project poses several challenges:
\begin{itemize}
  \item \textbf{Ambiguity in Language}: Responses often contain nuanced phrasing or implicit meanings 
  that make it difficult to determine focus or sentiment.
  \item \textbf{Overlapping Categories}: Some responses may include references to both individual and 
  team efforts, requiring careful textual interpretation.
\end{itemize}

Addressing these challenges will require careful preprocessing, robus labelling criteria and advanced
modelling techniques to ensure meaningful and reliable classifications.

\section{Task Definition}
This project involves a \textbf{multi-label classification task} where each data point is assigned two binary 
labels:

\begin{itemize}
  \item \textbf{Focus}: 2 classes (Individual, Team)
  \item \textbf{Sentiment}: 2 classes (Positive, Negative)
\end{itemize}

Sample annotations incldue ``Positive, Team'', ``Negative, Individual'', ``Positive, Individual'' and ``Negative, Team''.

\section{Data Source and Plan for Data Collection}
% May include how we are going to scrape the data and follow terms-of-service,
% API access and handling rate limiting, using open source data or any other 
% relevant details.
Our data will be interview transcripts on \href{https://asapsports.com/}{asapsports.com}, specifically the responses given by the interviewees 
(players and/or coaches). We will write an automated web scraping script in Python to collect the transcript text.\\

\indent There is neither a terms of service nor a Robots Exclusion Protocol (robots.txt) file on the website, likely 
because the website doesn’t get a lot of traffic. Since this is publicly available data taken from interviews, the 
biggest concern would be rate limiting. We would only need to scrape about 100 transcripts to meet our required number 
of data points, thus we can just limit the number of requests per second to not overload their servers.\\

\indent After tokenizing and removing unecessary information (e.g. interviewee name) we can store the corpus in a 
single file or multiple files for labelling purposes.

\section{Dataset Details}
\indent A data point will be considered a single paragraph of a response to an interview question by a player or coach. 
For each event, we will look at each and every day of interview recordings that are tracked. The events covered 
this time will be the entirety of both NBA and WNBA Finals, as well as both NBA and WNBA Drafts. Based on these 
events, there are a total of 17 days of interviews. Furthermore, each recording day has interviews from multiple 
different players and coaches that each answer questions with either a one or multi-paragraph response. Based on 
this information, our dataset is expected to contain approximately 2500 data points.
\newline \newline
\indent The following is a set of 3 data points taken from the Game 2 Postgame interview with Jason Kidd of the Dallas 
Mavericks:
\begin{enumerate}
    \item Yeah, we are not down. We're positive. This is a group that believes. We didn't get an opportunity to get a 
    split or win two here on the road. Now Boston held serve. Now we've got to go home and hold serve.
    \begin{itemize}
        \item \textbf{Assigned Labels: Positive, Team}
    \end{itemize}

    \item Big. The small things, you know, we have to do the small things, and that's part of the game. Those are points 
    that we left on the board, and we didn't shoot free throws well tonight, and we have to be better.
    \begin{itemize}
        \item \textbf{Assigned Labels: Negative, Team}
    \end{itemize}

    \item Yeah, I think Luka is a special player. He's one of, if not the best player in the world, and he causes a problem. 
    He's able to find guys. Again, creating open opportunities, and we just didn't take advantage of it.
    \begin{itemize}
        \item \textbf{Assigned Labels: Positive, Individual}
    \end{itemize}
\end{enumerate}

\section{Team Contract}
% Description of the expectations and/or roles that team members agree upon 
% for the semester. Part of final project grade will be based on whether 
% the teammates agree that we followed this contract. It should specify the
% overall purpose of the team, the responsibilities and the ground rules
% the team members agree to follow.
This section outlines the expected team dynamic and roles/responsibilities of each team member.

\subsection{Team Purpose}
Our team's purpose is to analyze and interpret communication dynamics
within sports contexts to gain insights into individual and
team-focused behaviors and sentiments. By classifying responses and
identifying patterns, we aim to enhance understanding of team
dynamics, promote collaboration, and support decision-making that
drives collective success. Through this, we contribute to fostering a
positive, cohesive environment where both individual strengths and
team goals are harmoniously aligned.

\subsection{Team Member Roles and Responsibilities}
This section details rules and expectations every team member agrees to.

\begin{itemize}
  \item Team members are expected to come to meetings prepared, having
    completed any previously assigned tasks or research. An agenda will
    be shared at the start of the
    meeting in the first five minutes.
  \item Attendance (either in person or virtual) is mandatory for all
    scheduled meetings and
    work periods unless a valid reason is provided. Active participation
    and engagement
    during the meeting is expected from everyone.
  \item Meetings will be held once a week for 60 (time limit open to extension based on each member's availability),
    with additional work periods scheduled as needed. Any changes to
    the schedule
    will be agreed upon by the team. Meetings will be held every Thursday
    from 2:30 to
    3:30 pm EST in person, unless agreed upon otherwise.
  \item The primary communication platform will be a Teams group chat. All team
    members are expected to check and respond to messages within 24
    hours during weekdays and 48 hours during weekends. Urgent
    matters should be flagged for immediate attention.
  \item All work submitted should meet the agreed-upon standards of
    quality and be
    completed on time. Team members are responsible for proofreading their
    work and ensuring it aligns with the project’s goals before submission.
  \item Decisions will be made by consensus whenever possible. If
    consensus cannot be
    reached, the team will vote on the issue. In case of a tie or
    unresolved conflict, the
    teaching assistant and/or the professor will have the final say after
    considering all opinions.
  \item Team members will communicate respectfully and
    professionally, both during
    meetings and in written correspondence. Constructive feedback is encouraged,
    and all opinions should be valued and considered.
\end{itemize}

\subsection{Facilitation Activities}
The team will adopt a collaborative and structured approach to
leadership, facilitation, and management activities to ensure smooth
operations and the achievement of objectives. Leadership roles will
be clearly defined, with a focus on fostering open communication,
delegation of responsibilities, and promoting accountability among
all team members.\\

Facilitation will involve regular meetings to align
on goals, review progress, and address challenges collectively,
ensuring everyone’s voice is heard and valued. Management activities
will include maintaining a clear workflow, tracking deliverables
through project management tools like GitHub, and encouraging
continuous feedback for improvement. The leadership approach will
emphasize adaptability, teamwork, and proactive problem-solving to
ensure the team remains aligned, motivated, and effective in
achieving its purpose.\\

For equal distirbution of work, the leadership role will be rotated among
all team members over the course of the semester ensuring, each team member is 
awarded the opportunity to be in the role once over the span of the project.

\subsection{Procedure for Meeting Unmet Expectations}
If the work cannot be completed to the team’s standard, the team
member responsible
must either provide a clear explanation or take steps to improve and
resubmit the work to
meet the required quality.
The steps outlined for raising and discussing concerns include:
\begin{enumerate}
  \item Defining the root of the problem and describing the conflict.
  \item Recognizing different viewpoints for all parties or areas
    that lack mutual understanding.
  \item Establishing a common goal.
  \item Developing a plan that both parties can agree on, addressing
    the main issue.
  \item Agreeing on a plan of action to reach an agreement.
  \item Monitoring and following up on the resolution
    progress to identify the actions leading up after the conflict.
\end{enumerate}

The leader and/or mediator must be an individual that is not involved
in the conflict. They will ensure that everyone’s voices are heard,
while making sure
that there is an equal
and balanced discussion. A space should be fostered where everyone’s
voices are heard and
respected in an equal manner that is fair to all. In addition, they
must be empathetic about
all parties involved, as well as understanding the situation, and
willing to support both
sides to work through the differences. Additionally, they must be
approachable and
respectful.
\\

\textbf{Responsibilities of Leader/Mediator:}
\begin{itemize}
  \item Coordinate meetings to discuss issues
  \item Keep discussions productive and respectful to all parties
  \item Document any key points and action items from the meetings
    (meeting minutes)
  \item Track progress of resolution and follow up with all parties
\end{itemize}

\subsection{Procedure for Handling Peer Feedback}
\begin{itemize}
  \item All feedback can be submitted directly to the team member
    concerned either privately or as a team.

  \item If there is a group concern, the whole team can hold a
    behavior review and
    address the concerns as a team.
  \item Feedback should be constructive, respectful, and actionable.
    Team members
    should encourage focusing on specific behaviors, outcomes, or
    suggestions for
    improvement.
  \item Ensure that feedback is objective, free of personal attacks,
    and focuses on
    actions, not personality.
  \item Ensure that feedback includes both positive aspects (what went well) and
    areas for growth.
  \item Under no circumstances should feedback be disrespectful or inappropriate
    to the person(s) receiving it. This includes but is not limited to, using
    derogatory or insulting language, making personal attacks, or dismissing the
    concerns of other members.
  \item All feedback should be in written format (jot notes or
    paragraphs) along with
    suggestions for improvement. Feedback should be submitted to the
    team members
    at the mid-project evaluation and the end of project evaluation.
  \item A team member receiving feedback should review said feedback and the
    suggestions for improvement, implementing any needed changes by team
    milestone check-ins and reviews.
  \item The team, through a brief meeting, will hold team members
    accountable for
    implementing their constructive feedback. Team members will
    support each other
    in accomplishing their goals and taking care to address and
    mitigate any concerns
    or challenges.
\end{itemize}

\newpage
\section{Team Contract Agreement}
All team members are responsible for performing the required work
outlined in each milestone honestly, without plagiarism and cheating.\\

This is a reminder that submitting this work with your name and macID is a statement 
and understanding that this work is your own and adheres to the Academic Integrity Policy 
of McMaster University. If you agree to terms above, please sign below in the designated 
area by printing your full name:
\begin{table}[H]
    \centering
    \begin{tabular}{|l|l|}
    \hline
    \textbf{macID} & \textbf{Full Name as Signature} \\ \hline
    fatimn8 & Nawaal Fatima          \\ \hline
    lij499  & Junnan Li                       \\ \hline
    bhuiyr2 &  Rashad Ahmed Bhuiyan                      \\ \hline
    gulats10 & Sumanya Gulati         \\ \hline
    \end{tabular}
    \caption{Team Contract Agreement Signatures}
    \label{tab:team-signatures}
    \end{table}

\end{document}