\documentclass[titlepage]{article}

\usepackage{tabularx}
\usepackage{booktabs}
\usepackage[bottom=3cm, right=3cm, left=3cm, top=3cm]{geometry}
\usepackage{graphicx}
\usepackage{hyperref}
\usepackage{float}
\newcommand\tab[1][1cm]{\hspace*{#1}}

\title{COMPSCI 4NL3: Natural Language Processing\\
Annotation Guidelines}

\author{Team 4\\
\\ Junnan Li
\\ Nawaal Fatima
\\ Rashad Bhuiyan
\\ Sumanya Gulati}                  

\date{02 February 2025}

\begin{document}

\begin{titlepage}
  \maketitle
\end{titlepage}

\newpage

\section{Annotation Guidelines}
Below are the guidelines developed for the annotation process and to be agreed upon 
by the annotators. Note that these guidelines have been specifically developed for 
annotating interview responses from \href{https://asapsports.com/}{asapsports} and should not be directly used to
annotate other types of text before considering and adapting to relevant contexts and media. 
It must also be noted that all examples provided here are related to NBA or WNBA interview responses.

\section{Accessing the Dataset}
To access the dataset, the annotator is asked to navigate to the \textbf{annotations} folder which can be
found in the same directory as the annotation guidelines document. Here, you will find 8 CSV files labelled as
interview1\_annotations.csv, interview2\_annotations.csv and so on till interview8\_annotations.csv, along with a 
markdown file titled README.md. \\

The markdown file contains detailed information about how the dataset was curated and how the steps can be 
replicated by another user, if desired. To begin annotating, each annotator must open the specific CSV file using 
MS Excel (recommended) or Google Sheets.

\section{Labels and Defintions}
The labels are divided into the classes - \textbf{focus} and \textbf{sentiment}. Focus can be either of 
the two labels - \textbf{individual} or \textbf{team} and similarly, sentiment can be either \textbf{positive} or \textbf{negative}. 
Focus has to do with whether the response focuses on the \emph{individual (self)} or the \emph{team} and 
sentiment is based on whether the speaker is expressing \emph{positive} or \emph{negative} emotion.

\section{Rules}
For annotating focus, think about whether the coach/player emphasizes teamwork or individual contributions. If both are labels are applicable,
choose the one that seems more appropriate (has a stronger inference) and if both of carry the same inference, select
the label whose instance occurs first . \\

For annotating sentiment, think about the tone coach/player is trying to convey. If they are excited, hopeful, or showing a 
growth-mindset this would be considered positive. However, if the response is lacking confidence or conveys disappointment, 
this would be labelled negative. \\

When annotating, each annotator is expected to mark the appropriate label for each class with a \textbf{1} or 
a \textbf{Yes} in that respective column for each datapoint (or row). Please note that each class must have 
\textbf{only 1} label. This means, there are 4 possible combinations for each datapoint:
\begin{itemize}
    \item (Team, Positive)
    \item (Team, Negative)
    \item (Individual, Positive)
    \item (Individual, Negative)
\end{itemize}

\section{Case Examples}
\begin{enumerate}
    \item Yeah, we are not down. We're positive. This is a group that believes. We didn't get an opportunity to get a 
    split or win two here on the road. Now Boston held serve. Now we've got to go home and hold serve.
    \begin{itemize}
        \item \textbf{Assigned Labels: Positive, Team}
    \end{itemize}
    This has been labelled positive as the overall sentiment is hopeful. They mention that they are not down and that this is 
    a group that believes in themselves so they are determined to win. 'We' has been used multiple times when talking about 
    their attitude so this is a sign of a team-focused mentality.

    \item Big. The small things, you know, we have to do the small things, and that's part of the game. Those are points 
    that we left on the board, and we didn't shoot free throws well tonight, and we have to be better.
    \begin{itemize}
        \item \textbf{Assigned Labels: Negative, Team}
    \end{itemize}
    This has been labelled negative as they say mention how they didn't do well which cost them points on the board. It shows 
    disapointment and a desire to improve. It is also team-focused as they are talking about the entire team as a collective 
    and not any single player.

    \item Yeah, I think Luka is a special player. He's one of, if not the best player in the world, and he causes a problem. 
    He's able to find guys. Again, creating open opportunities, and we just didn't take advantage of it.
    \begin{itemize}
        \item \textbf{Assigned Labels: Positive, Individual}
    \end{itemize}
    This has been labelled positive as they are complimenting a player. The comments are mainly focused on a singular player 
    and how they played thus it has been labelled individual.
\end{enumerate}

\section{Contact Information}
Please contact one of the following members of Team 4 for additional information and issue resolution:
\begin{itemize}
    \item Junnan Li: \textbf{lij499@mcmaster.ca}
    \item Nawaal Fatima: \textbf{fatimn8@mcmaster.ca}
    \item Rashad Bhuiyan: \textbf{bhuiyr2@mcmaster.ca}
    \item Sumanya Gulati: \textbf{gulats10@mcmaster.ca}
\end{itemize}

\end{document}